%%%%%%%%%%%%%%%%%%%%%%%%%%%%%%%%%%%%%%%%%
% Short Sectioned Assignment
% LaTeX Template
% Version 1.0 (5/5/12)
%
% This template has been downloaded from:
% http://www.LaTeXTemplates.com
%
% Original author:
% Frits Wenneker (http://www.howtotex.com)
%
% License:
% CC BY-NC-SA 3.0 (http://creativecommons.org/licenses/by-nc-sa/3.0/)
%
%%%%%%%%%%%%%%%%%%%%%%%%%%%%%%%%%%%%%%%%%

%----------------------------------------------------------------------------------------
%	PACKAGES AND OTHER DOCUMENT CONFIGURATIONS
%----------------------------------------------------------------------------------------

\documentclass[paper=a4, fontsize=10pt]{scrartcl} % A4 paper and 11pt font size

\usepackage[T1]{fontenc} % Use 8-bit encoding that has 256 glyphs
\usepackage{fourier} % Use the Adobe Utopia font for the document - comment this line to return to the LaTeX default
\usepackage[english]{babel} % English language/hyphenation
\usepackage{amsmath,amsfonts,amsthm} % Math packages
\usepackage{color, soul}

\usepackage{lipsum} % Used for inserting dummy 'Lorem ipsum' text into the template

\usepackage{sectsty} % Allows customizing section commands
\allsectionsfont{\centering \normalfont\scshape} % Make all sections centered, the default font and small caps

\usepackage{fancyhdr} % Custom headers and footers
\pagestyle{fancyplain} % Makes all pages in the document conform to the custom headers and footers
\fancyhead{} % No page header - if you want one, create it in the same way as the footers below
\fancyfoot[L]{} % Empty left footer
\fancyfoot[C]{} % Empty center footer
\fancyfoot[R]{\thepage} % Page numbering for right footer
\renewcommand{\headrulewidth}{0pt} % Remove header underlines
\renewcommand{\footrulewidth}{0pt} % Remove footer underlines
\setlength{\headheight}{13.6pt} % Customize the height of the header

%\numberwithin{equation}{section} % Number equations within sections (i.e. 1.1, 1.2, 2.1, 2.2 instead of 1, 2, 3, 4)
%\numberwithin{figure}{section} % Number figures within sections (i.e. 1.1, 1.2, 2.1, 2.2 instead of 1, 2, 3, 4)
%\numberwithin{table}{section} % Number tables within sections (i.e. 1.1, 1.2, 2.1, 2.2 instead of 1, 2, 3, 4)

\setlength\parindent{0pt} % Removes all indentation from paragraphs - comment this line for an assignment with lots of text

%----------------------------------------------------------------------------------------
%	TITLE SECTION
%----------------------------------------------------------------------------------------

\newcommand{\horrule}[1]{\rule{\linewidth}{#1}} % Create horizontal rule command with 1 argument of height

\title{
\normalfont \normalsize
\textsc{Utah State University} \\ [25pt] % Your university, school and/or department name(s)
\horrule{0.5pt} \\[0.4cm] % Thin top horizontal rule
\huge Homework 2 \\ % The assignment title
\horrule{2pt} \\[0.5cm] % Thick bottom horizontal rule
}

\author{Jonathan Petersen} % Your name

\date{\normalsize\today} % Today's date or a custom date

\begin{document}

\maketitle % Print the title

\section*{Chapter 2.1}
%----------------------------------------------------------------------------------------
%	PROBLEM
%----------------------------------------------------------------------------------------
\subsection*{}
5. The first of these equations plus the second equals the third:

\begin{align*}
 x +  y +  z &= 2 \\
 x + 2y +  z &= 3 \\
2x + 3y + 2z &= 5 \\
\end{align*}

The first two planes meet along a line. The third plane contains the line, because if $x, y, z$ satisfy the first two equations then they also \hl{satisfy the third.} The equations have infinitely many solutions (The whole line L). Find three solutions on L.


\begin{alignat}{6}
x &= 2  \qquad &y &= 1  \qquad &z &= -1 \\
x &= 5  \qquad &y &= 1  \qquad &z &= -4 \\
x &= 10 \qquad &y &= 1  \qquad &z &= -9
\end{alignat}

%------------------------------------------------
\subsection*{}
6. Move the third plane in problem 5 to a parallel plane 2x + 3y + 2z = 9. now the three equations have no solution--why not? The first two planes meet along the line L, but the third plane doesn't \hl{contain or intersect} that line.

%------------------------------------------------
\subsection*{}

7. In problem five the columns are (1, 1, 2) and (1, 2, 3) and (1, 1, 2). This is a ``singular case'' because the third column is \hl{identical to the first.} Find two combinations of the columns that give b = (2, 3, 5).

\setcounter{equation}{0}
\begin{align}
	2
	\begin{bmatrix}
	1 \\ 1 \\ 2
	\end{bmatrix}
	+ 1
	\begin{bmatrix}
	1 \\ 2 \\ 3
	\end{bmatrix}
	- 1
	\begin{bmatrix}
	1 \\ 1 \\ 2
	\end{bmatrix}
	&=
	\begin{bmatrix}
	2 \\ 3 \\ 5
	\end{bmatrix}
	\\
	5
	\begin{bmatrix}
	1 \\ 1 \\ 2
	\end{bmatrix}
	+ 1
	\begin{bmatrix}
	1 \\ 2 \\ 3
	\end{bmatrix}
	- 4
	\begin{bmatrix}
	1 \\ 1 \\ 2
	\end{bmatrix}
	&=
	\begin{bmatrix}
	2 \\ 3 \\ 5
	\end{bmatrix}
\end{align}

This is only possible for b = (4, 6, c) if c = \hl{10.}

%------------------------------------------------
\subsection*{}

9. Compute each Ax by dot products of the rows with the column vector:

	\setcounter{equation}{0}
	\begin{align*}
		\begin{bmatrix}
			  1 & 2 & 4 \\
			 -2 & 3 & 1 \\
			 -4 & 1 & 2 \\
		\end{bmatrix}
		\begin{bmatrix}
		2 \\
		2 \\
		3 \\
		\end{bmatrix}
	\end{align*}

	\begin{align*}
	2 \times  1 + 2 \times 2 + 3 \times 4  &= \\  2 + 4 + 12  &=  18 \\
	2 \times -2 + 2 \times 3 + 3 \times 1  &= \\ -4 + 6 + 3   &=  5 \\
	2 \times -4 + 2 \times 1 + 3 \times 2  &= \\ -8 + 2 + 6   &=  0
	\end{align*}

	\begin{align}
		Ax &=
		\begin{bmatrix}
		    18 \\
			 5 \\
	   		 0
	   	\end{bmatrix}
	\end{align}

	\begin{align*}
		\begin{bmatrix}
			2 & 1 & 0 & 0 \\
			1 & 2 & 1 & 0 \\
			0 & 1 & 2 & 1 \\
			0 & 0 & 1 & 2
		\end{bmatrix}
		=
		\begin{bmatrix}
			1 \\
			1 \\
			1 \\
			2
		\end{bmatrix}
	\end{align*}

	\begin{align*}
		1 \times 2 + 1 \times 1 + 1 \times 0 + 2 \times 0  &= \\  2 + 1 + 0 + 0  &=  3 \\
		1 \times 1 + 1 \times 2 + 1 \times 1 + 2 \times 0  &= \\  1 + 2 + 1 + 0  &=  4 \\
		1 \times 0 + 1 \times 1 + 1 \times 2 + 2 \times 1  &= \\  0 + 1 + 2 + 2  &=  5 \\
		1 \times 0 + 1 \times 0 + 1 \times 1 + 2 \times 2  &= \\  0 + 0 + 1 + 4  &=  5
	\end{align*}

	\begin{align*}
		Ax =
		\begin{bmatrix}
			3 \\
			4 \\
			5 \\
			5
		\end{bmatrix}
	\end{align*}

%------------------------------------------------
\subsection*{}
\setcounter{equation}{0}
10. Compute each Ax in problem 9 as a combination of the columns:

	\begin{align*}
		\begin{bmatrix}
			  1 & 2 & 4 \\
			 -2 & 3 & 1 \\
			 -4 & 1 & 2
 		\end{bmatrix}
 		\begin{bmatrix}
			 2 \\
			 2 \\
			 3 \\
 		\end{bmatrix}
	\end{align*}

	\begin{align*}
		2
		\begin{bmatrix}
			1 \\ -2 \\ -4
		\end{bmatrix}
		+ 2
		\begin{bmatrix}
			2 \\ 3 \\ 1
		\end{bmatrix}
		+ 3
		\begin{bmatrix}
			4 \\ 1 \\ 2
		\end{bmatrix}
		&= \\
		\begin{bmatrix}
			2 \\ -4 \\ -8
		\end{bmatrix}
		+
		\begin{bmatrix}
			4 \\ 6 \\ 2
		\end{bmatrix}
		+
		\begin{bmatrix}
			12 \\ 3 \\ 6
		\end{bmatrix}
		&=
		\begin{bmatrix}
			18 \\ 5 \\ 0
		\end{bmatrix}
	\end{align*}

	\begin{align}
		Ax =
		\begin{bmatrix}
			18 \\ 5 \\ 0
		\end{bmatrix}
	\end{align}

	\begin{align*}
		\begin{bmatrix}
			2 & 1 & 0 & 0 \\
			1 & 2 & 1 & 0 \\
			0 & 1 & 2 & 1 \\
			0 & 0 & 1 & 2
		\end{bmatrix}
		=
		\begin{bmatrix}
			1 \\
			1 \\
			1 \\
			2
		\end{bmatrix}
	\end{align*}

	\begin{align*}
		1
		\begin{bmatrix}
			2 \\
			1 \\
			0 \\
			0
		\end{bmatrix}
		+ 1
		\begin{bmatrix}
			1 \\
			2 \\
			1 \\
			0
		\end{bmatrix}
		+ 1
		\begin{bmatrix}
 			0 \\
 			1 \\
 			2 \\
 			1
 		\end{bmatrix}
		+ 2
		\begin{bmatrix}
 			0 \\
 			0 \\
 			1 \\
 			2
 		\end{bmatrix}
 		&= \\
 		\begin{bmatrix}
 			2 \\ 1 \\ 0 \\ 0
 		\end{bmatrix}
 		+
 		\begin{bmatrix}
 			1 \\ 2 \\ 1 \\ 0
 		\end{bmatrix}
 		+
 		\begin{bmatrix}
 			0 \\ 1 \\ 2 \\ 1
 		\end{bmatrix}
 		+
 		\begin{bmatrix}
 			0 \\ 0 \\ 2 \\ 4
 		\end{bmatrix}
 		&=
 		\begin{bmatrix}
 			3 \\ 4 \\ 5 \\ 5
 		\end{bmatrix}
 	\end{align*}

 	\begin{align}
 		Ax =
 		\begin{bmatrix}
 			3 \\ 4 \\ 5 \\ 5
 		\end{bmatrix}
 	\end{align}

%------------------------------------------------
\subsection*{}
\setcounter{equation}{0}
11. Find the two components of Ax by rows or by columns:

	\begin{align*}
		\begin{bmatrix}
			2 & 3 \\
			5 & 1
		\end{bmatrix}
		\begin{bmatrix}
			4 \\
			2
		\end{bmatrix}
	\end{align*}

	\begin{align*}
		4 \begin{bmatrix}
			2 \\ 5
		\end{bmatrix}
		+ 2 \begin{bmatrix}
			3 \\ 1
		\end{bmatrix}
		&= \\
		\begin{bmatrix}
			8 + 6 \\ 20 + 2
		\end{bmatrix}
		&=
		\begin{bmatrix}
			14 \\ 22
		\end{bmatrix}
	\end{align*}

	\begin{align}
		Ax &= \begin{bmatrix}
			14 \\ 22
		\end{bmatrix}
	\end{align}

	    and

	\begin{align*}
		\begin{bmatrix}
			3 & 6 \\
			6 & 12
		\end{bmatrix}
		\begin{bmatrix}
			2 \\
			-1
		\end{bmatrix}
	\end{align*}

	\begin{align*}
		2 \begin{bmatrix}
			3 \\ 6
		\end{bmatrix}
		- 1 \begin{bmatrix}
			6 \\ 12
		\end{bmatrix}
		&= \\
		\begin{bmatrix}
			6 - 6 \\ 12 - 12
		\end{bmatrix}
		&=
		\begin{bmatrix}
			0 \\ 0
		\end{bmatrix}
	\end{align*}

	\begin{align}
		Ax &= \begin{bmatrix}
			0 \\ 0
		\end{bmatrix}
	\end{align}

	    and

	\begin{align*}
		\begin{bmatrix}
			1 & 2 & 4 \\
			2 & 0 & 1
		\end{bmatrix}
		\begin{bmatrix}
			3 \\
			1 \\
			1
		\end{bmatrix}
	\end{align*}

	\begin{align*}
		3 \begin{bmatrix}
			1 \\ 2
		\end{bmatrix}
		+ 1 \begin{bmatrix}
			2 \\ 0
		\end{bmatrix}
		+ 1 \begin{bmatrix}
			4 \\ 1
		\end{bmatrix}
		&= \\
		\begin{bmatrix}
			3 + 2 + 4 \\ 6 + 0 + 1
		\end{bmatrix}
		&=
		\begin{bmatrix}
			9 \\ 7
		\end{bmatrix}
	\end{align*}

	\begin{align}
		Ax &= \begin{bmatrix}
			9 \\ 7
		\end{bmatrix}
	\end{align}

%------------------------------------------------

\section*{Chapter 2.2}

%------------------------------------------------
\subsection*{}
\setcounter{equation}{0}
4. What multiple $l$ of equation 1 should be subtracted from equation 2 to remove $c$?

	\begin{align}
		ax + by &= f \\
		cx + dy &= g
	\end{align}

	Equation 1 should be divided by $a$ and multiplied by $c$, or, in other words \dfrac{c}{a}.

%------------------------------------------------
\subsection*{}
\setcounter{equation}{0}

6. Choose a coefficient $b$ that makes this system singular. Then choose a right side $g$ that makes it solvable.

	\begin{align*}
		2x + by &= 16 \\
		4x + 8y &= g
	\end{align*}

	If $b = 4$ then this system is singluar, as that value of $b$ would make the line describing the first equation parallel to the line describing the second.

	If $g = 32$ then this singular system is solvable, as the first and second equations describe the same line, so every point on the line is a solution.

Find two solutions in that singular case.

	\begin{alignat}{4}
		x &= 2 &\qquad y &= 3 \\
		x &= 4 &\qquad y &= 2
	\end{alignat}

%------------------------------------------------
\subsection*{}
\setcounter{equation}{0}

13. Apply elimination (circle the pivots) and back substitution to solve:

	\begin{align*}
		2x - 3y &= 3 \\
		4x - 5y + z &= 7 \\
		2x - y - 3z &= 5
	\end{align*}

	\begin{align*}
		A =
		\begin{bmatrix}
			2 & -3 & 0 \\
			4 & -5 & 1 \\
			2 & -1 & -3
		\end{bmatrix}
		&=
		\begin{bmatrix}
			3 \\ 7 \\ 5
		\end{bmatrix}
		\\
		\begin{bmatrix}
			2 & -3 & 0 &| 3\\
			4 & -5 & 1 &| 7\\
			2 & -1 & -3 &| 5
		\end{bmatrix}
		\begin{bmatrix}
			1 & 0 & 0 \\
			0 & 0 & 1 \\
			0 & 1 & 0
		\end{bmatrix}
		&=
		\begin{bmatrix}
			2 & -3 & 0 &| 3 \\
			2 & -1 & -3 &| 5 \\
			4 & -5 & 1 &| 7
		\end{bmatrix}
		\\
		\begin{bmatrix}
			2 & -3 & 0 &| 3 \\
			2 & -1 & -3 &| 5 \\
			4 & -5 & 1 &| 7
		\end{bmatrix}
		\begin{bmatrix}
			1 & 0 & 0 \\
			-1 & 1 & 0 \\
			0 & 0 & 1
		\end{bmatrix}
		&=
		\begin{bmatrix}
			2 & -3 & 0 &| 3 \\
			0 & 2 & -3 &| 2 \\
			4 & -5 & 1 &| 7
		\end{bmatrix}
		\\
		\begin{bmatrix}
			2 & -3 & 0 &| 3 \\
			0 & 2 & -3 &| 2 \\
			4 & -5 & 1 &| 7
		\end{bmatrix}
		\begin{bmatrix}
			1 & 0 & 0 \\
			0 & 1 & 0 \\
			-2 & 0 & 1
		\end{bmatrix}
		&=
		\begin{bmatrix}
			2 & -3 & 0 &| 3 \\
			0 & 2 & -3 &| 2 \\
			0 & 1 & 1 &| 1
		\end{bmatrix}
		\\
		\begin{bmatrix}
			2 & -3 & 0 &| 3 \\
			0 & 2 & -3 &| 2 \\
			0 & 1 & 1 &| 1
		\end{bmatrix}
		\begin{bmatrix}
			1 & 0 & 0 \\
			0 & 0 & 1 \\
			0 & 1 & 0
		\end{bmatrix}
		&=
		\begin{bmatrix}
			2 & -3 & 0 &| 3 \\
			0 & 1 & 1 &| 1 \\
			0 & 2 & -3 &| 2
		\end{bmatrix}
		\\
		\begin{bmatrix}
			2 & -3 & 0 &| 3 \\
			0 & 1 & 1 &| 1 \\
			0 & 2 & -3 &| 2
		\end{bmatrix}
		\begin{bmatrix}
			1 & 0 & 0 \\
			0 & 1 & 0 \\
			0 & -2 & 1
		\end{bmatrix}
		&=
		\begin{bmatrix}
			2 & -3 & 0 &| 3 \\
			0 & 1 & 1 &| 1 \\
			0 & 0 & -5 &| -4
		\end{bmatrix}
		\\
		\begin{bmatrix}
			2 & -3 & 0 \\
			0 & 1 & 1 \\
			0 & 0 & -5
		\end{bmatrix}
		&=
		\begin{bmatrix}
			3 \\ 1 \\ -4
		\end{bmatrix}
	\end{align*}

	Therefore,

	\begin{align*}
		A &= \qquad
		\begin{matrix}
			2x & -3y &     &= 3 \\
			   &   y &   z &= 1 \\
			   &     & -5z &= -4
		\end{matrix}
		\\
		z &= \frac{4}{5} \\
		\\
		y + z &= 1 \\
		y + \frac{4}{5} &= 1 \\
		y &= \frac{5}{4} \\
		\\
		2x -3y &= 3 \\
		2x -3 \times \frac{5}{4} &= 3 \\
		2x -\frac{15}{4} &= 3 \\
		2x &= \frac{12}{4} + \frac{15}{4} \\
		x &= 2 \times \frac{27}{4} \\
		x &= \frac{27}{2}
	\end{align*}

List three row operations: \\
Subtract 1 times row 1 from row 2. \\
Subtract 2 times row 1 from row 3. \\
Subtract 2 times row 2 from row 3. \\

%------------------------------------------------
\subsection*{}
\setcounter{equation}{0}

14. Which number b leads later to a row exchange? Which b leads to a missing pivot? In that singular case find a nonzero solution x, y, z.


	\begin{align*}
		\begin{matrix}
			x &+ by &    &= 0 \\
			x &- 2y &- z &= 0 \\
			  &   y &+ z &= 0
		\end{matrix}
	\end{align*}


$b = -1$ later leads to a row exchange, since row 2 cannot be used to remove the y in row three. \\
$b = -2$ leads to a missing pivot, as any change to remove x makes y in row 2 zero. \\
\\
Suppose $b = -2$, the matrix is then \\

\begin{align*}
	\begin{bmatrix}
		1 & -2 &  0 &| 0 \\
		1 & -2 & -1 &| 0 \\
		0 &  1 &  1 &| 0
	\end{bmatrix}
\end{align*}

Therefore: \\

\begin{align*}
	x - 2y &= 0 \\
	-z &= 0 \\
	y + z &= 0
\end{align*}

In this case, (0, 0, 0) is the only solution.

\end{document}
